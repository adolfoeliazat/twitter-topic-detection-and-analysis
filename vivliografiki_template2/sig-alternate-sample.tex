% This is "sig-alternate.tex" V2.1 April 2013
% This file should be compiled with V2.5 of "sig-alternate.cls" May 2012
%
% This example file demonstrates the use of the 'sig-alternate.cls'
% V2.5 LaTeX2e document class file. It is for those submitting
% articles to ACM Conference Proceedings WHO DO NOT WISH TO
% STRICTLY ADHERE TO THE SIGS (PUBS-BOARD-ENDORSED) STYLE.
% The 'sig-alternate.cls' file will produce a similar-looking,
% albeit, 'tighter' paper resulting in, invariably, fewer pages.
%
% ----------------------------------------------------------------------------------------------------------------
% This .tex file (and associated .cls V2.5) produces:
%       1) The Permission Statement
%       2) The Conference (location) Info information
%       3) The Copyright Line with ACM data
%       4) NO page numbers
%
% as against the acm_proc_article-sp.cls file which
% DOES NOT produce 1) thru' 3) above.
%
% Using 'sig-alternate.cls' you have control, however, from within
% the source .tex file, over both the CopyrightYear
% (defaulted to 200X) and the ACM Copyright Data
% (defaulted to X-XXXXX-XX-X/XX/XX).
% e.g.
% \CopyrightYear{2007} will cause 2007 to appear in the copyright line.
% \crdata{0-12345-67-8/90/12} will cause 0-12345-67-8/90/12 to appear in the copyright line.
%
% ---------------------------------------------------------------------------------------------------------------
% This .tex source is an example which *does* use
% the .bib file (from which the .bbl file % is produced).
% REMEMBER HOWEVER: After having produced the .bbl file,
% and prior to final submission, you *NEED* to 'insert'
% your .bbl file into your source .tex file so as to provide
% ONE 'self-contained' source file.
%
% ================= IF YOU HAVE QUESTIONS =======================
% Questions regarding the SIGS styles, SIGS policies and
% procedures, Conferences etc. should be sent to
% Adrienne Griscti (griscti@acm.org)
%
% Technical questions _only_ to
% Gerald Murray (murray@hq.acm.org)
% ===============================================================
%
% For tracking purposes - this is V2.0 - May 2012

\documentclass{sig-alternate-05-2015}


\begin{document}

% Copyright
\setcopyright{acmcopyright}
%\setcopyright{acmlicensed}
%\setcopyright{rightsretained}
%\setcopyright{usgov}
%\setcopyright{usgovmixed}
%\setcopyright{cagov}
%\setcopyright{cagovmixed}


% DOI
\doi{10.475/123_4}

% ISBN
\isbn{123-4567-24-567/08/06}

%Conference
\conferenceinfo{PLDI '13}{June 16--19, 2013, Seattle, WA, USA}

\acmPrice{\$15.00}

%
% --- Author Metadata here ---
\conferenceinfo{WOODSTOCK}{'97 El Paso, Texas USA}
%\CopyrightYear{2007} % Allows default copyright year (20XX) to be over-ridden - IF NEED BE.
%\crdata{0-12345-67-8/90/01}  % Allows default copyright data (0-89791-88-6/97/05) to be over-ridden - IF NEED BE.
% --- End of Author Metadata ---

\title{Alternate {\ttlit ACM} SIG Proceedings Paper in LaTeX
Format\titlenote{(Produces the permission block, and
copyright information). For use with
SIG-ALTERNATE.CLS. Supported by ACM.}}
\subtitle{[Extended Abstract]
\titlenote{A full version of this paper is available as
\textit{Author's Guide to Preparing ACM SIG Proceedings Using
\LaTeX$2_\epsilon$\ and BibTeX} at
\texttt{www.acm.org/eaddress.htm}}}
%
% You need the command \numberofauthors to handle the 'placement
% and alignment' of the authors beneath the title.
%
% For aesthetic reasons, we recommend 'three authors at a time'
% i.e. three 'name/affiliation blocks' be placed beneath the title.
%
% NOTE: You are NOT restricted in how many 'rows' of
% "name/affiliations" may appear. We just ask that you restrict
% the number of 'columns' to three.
%
% Because of the available 'opening page real-estate'
% we ask you to refrain from putting more than six authors
% (two rows with three columns) beneath the article title.
% More than six makes the first-page appear very cluttered indeed.
%
% Use the \alignauthor commands to handle the names
% and affiliations for an 'aesthetic maximum' of six authors.
% Add names, affiliations, addresses for
% the seventh etc. author(s) as the argument for the
% \additionalauthors command.
% These 'additional authors' will be output/set for you
% without further effort on your part as the last section in
% the body of your article BEFORE References or any Appendices.

\numberofauthors{8} %  in this sample file, there are a *total*
% of EIGHT authors. SIX appear on the 'first-page' (for formatting
% reasons) and the remaining two appear in the \additionalauthors section.
%
\author{
% You can go ahead and credit any number of authors here,
% e.g. one 'row of three' or two rows (consisting of one row of three
% and a second row of one, two or three).
%
% The command \alignauthor (no curly braces needed) should
% precede each author name, affiliation/snail-mail address and
% e-mail address. Additionally, tag each line of
% affiliation/address with \affaddr, and tag the
% e-mail address with \email.
%
% 1st. author
\alignauthor
Ben Trovato\titlenote{Dr.~Trovato insisted his name be first.}\\
       \affaddr{Institute for Clarity in Documentation}\\
       \affaddr{1932 Wallamaloo Lane}\\
       \affaddr{Wallamaloo, New Zealand}\\
       \email{trovato@corporation.com}
% 2nd. author
\alignauthor
G.K.M. Tobin\titlenote{The secretary disavows
any knowledge of this author's actions.}\\
       \affaddr{Institute for Clarity in Documentation}\\
       \affaddr{P.O. Box 1212}\\
       \affaddr{Dublin, Ohio 43017-6221}\\
       \email{webmaster@marysville-ohio.com}
% 3rd. author
\alignauthor Lars Th{\o}rv{\"a}ld\titlenote{This author is the
one who did all the really hard work.}\\
       \affaddr{The Th{\o}rv{\"a}ld Group}\\
       \affaddr{1 Th{\o}rv{\"a}ld Circle}\\
       \affaddr{Hekla, Iceland}\\
       \email{larst@affiliation.org}
\and  % use '\and' if you need 'another row' of author names
% 4th. author
\alignauthor Lawrence P. Leipuner\\
       \affaddr{Brookhaven Laboratories}\\
       \affaddr{Brookhaven National Lab}\\
       \affaddr{P.O. Box 5000}\\
       \email{lleipuner@researchlabs.org}
% 5th. author
\alignauthor Sean Fogarty\\
       \affaddr{NASA Ames Research Center}\\
       \affaddr{Moffett Field}\\
       \affaddr{California 94035}\\
       \email{fogartys@amesres.org}
% 6th. author
\alignauthor Charles Palmer\\
       \affaddr{Palmer Research Laboratories}\\
       \affaddr{8600 Datapoint Drive}\\
       \affaddr{San Antonio, Texas 78229}\\
       \email{cpalmer@prl.com}
}
% There's nothing stopping you putting the seventh, eighth, etc.
% author on the opening page (as the 'third row') but we ask,
% for aesthetic reasons that you place these 'additional authors'
% in the \additional authors block, viz.
\additionalauthors{Additional authors: John Smith (The Th{\o}rv{\"a}ld Group,
email: {\texttt{jsmith@affiliation.org}}) and Julius P.~Kumquat
(The Kumquat Consortium, email: {\texttt{jpkumquat@consortium.net}}).}
\date{30 July 1999}
% Just remember to make sure that the TOTAL number of authors
% is the number that will appear on the first page PLUS the
% number that will appear in the \additionalauthors section.

\maketitle
\begin{abstract}
todo
\end{abstract}


%
% The code below should be generated by the tool at
% http://dl.acm.org/ccs.cfm
% Please copy and paste the code instead of the example below. 
%
%\begin{}
%<ccs2012>
 %<concept>
  %<concept_id>10010520.10010553.10010562</concept_id>
  %<concept_desc>Computer systems organization~Embedded systems</concept_desc>
  %<concept_significance>500</concept_significance>
 %</concept>
 %<concept>
  %<concept_id>10010520.10010575.10010755</concept_id>
  %<concept_desc>Computer systems organization~Redundancy</concept_desc>
  %<concept_significance>300</concept_significance>
 %</concept>
 %<concept>
  %<concept_id>10010520.10010553.10010554</concept_id>
  %<concept_desc>Computer systems organization~Robotics</concept_desc>
  %<concept_significance>100</concept_significance>
 %</concept>
 %<concept>
  %<concept_id>10003033.10003083.10003095</concept_id>
  %<concept_desc>Networks~Network reliability</concept_desc>
  %<concept_significance>100</concept_significance>
 %</concept>
%</ccs2012>  
%\end{}

\ccsdesc[500]{Computer systems organization~Embedded systems}
\ccsdesc[300]{Computer systems organization~Redundancy}
\ccsdesc{Computer systems organization~Robotics}
\ccsdesc[100]{Networks~Network reliability}


%
% End generated code
%

%
%  Use this command to print the description
%
\printccsdesc

% We no longer use \terms command
%\terms{Theory}

\keywords{ACM proceedings; \LaTeX; text tagging}

\section{Introduction}
Nowadays, the World Wide Web has transformed from a large, static library that people only browse into a vast and dynamic information resource. Relying on this, social networks is a very popular and powerful tool for expressing opinions, broadcasting news, and simply communicating with friends. People using them for commenting on significant events in real time, with several hundred micro-blogs posted each second. 

The most popular micro-blogging service is Twitter. The popularity of Twitter stems from its availability on a number of different electronic devices (web and cell phones. There is a prevalence of a subculture in Twitter that encourages users to acquire a large friend pool, as well as send tweets on a wide variety of subjects, typically several times a day. 

Monitoring and analysing this rich and continuous flow of user-generated content can yield unprecedentedly valuable information, which would not have been available from traditional
media outlets. Tweets can be seen as a dynamic source of information enabling individuals, corporations, and government organizations to stay informed of “what is happening
now.” For instance, people would be interested in getting advice, opinions, facts, or updates on news or events. Companies are increasingly using Twitter to advertise and recommend products, brands, and services; to build and maintain reputations; to analyse users’ sentiment regarding their products (or those of their competitors); to respond to customers’ complaints; and to improve decision making and business intelligence. Twitter has also emerged as a fast communication channel for gathering and spreading breaking news, for predicting election results, and for sharing political events and conversations. It has also become an important analytical tool for crime prediction and monitoring terrorist activities.

Twitter promotes an attractive style stating breaking news, as there is very little lag between the time that an event happens or is first reported in the news media and the time at which it is the subject of a posting on Twitter. Twitter can be characterized as an endless database, which collects millions of real-time short text messages every second. Tweets also have a mechanism by which the user can link to other objects on the web such as articles, images or videos which is typically used to link tweets to related material on the Internet. Thereafter, the first result is that the size of information is multiplied and the variety of references is bigger, as well. These messages are not only just data, but they can be manipulated efficiently. One well-timed subject of research is to use those messages for event detection. In other words, the tweets is a source of inventing which topics are more seasonable. Event detection has also instant impact on the world, through the quick transmission of the news and necessary briefing in some cases. 

With the passage of time and the effect of more and more users the topic acquire much popularity. Through this phenomenon we can form a general summarization of the event. This process is called event summarization. The massive crowd keeps close pace with the development of trending topics and provide the timely updated information. Twitter has shown its powerful ability in information delivery in many events, like the wildfires in San Diego and the earthquake in Japan. In response to searches for ongoing events, today's major search engines simply find tweets that match the query terms, and present the most recent ones. This approach has the advantage of leveraging existing query matching technologies, and for simple one-shot events such as earthquakes it works well. However, for events that have "structure" or are long-running, and where users are likely to want a summary of all occurrences so far, this approach is often unsatisfactory.

Event detection is a growing domain of research. Many different species of algorithms have been detected regarding this sector. A common approach are techniques that are based on text categorization. In addition, there are some methods that reclaim the display frequency of each term.

The computational treatment of sentiment has recently attracted a great deal of attention, in part because of its potential applications. One of the main reasons for sentiment analysis is the aforementioned increase of user-generated content on the Web which has resulted in a wealth of information that is potentially of vital importance to institutions and companies. Typically, document-based sentiment analysis processes operate at a particular level, i.e. at the word or sentence level, for extracting a document's sentiment. In machine learning, the most popular approach for sentiment analysis, the selection of appropriate features for representing a document is crucial. In sentiment identification at the word level different types of features have been introduced, which are either sentiment-based (e.g. words which express a specific sentiment), syntactic-based (e.g. part-of-speech and n-grams), or semantic-based (e.g. semantic word vector spaces which capture the meaning of each word).

Document-level polarity classification is not a special case of text categorization with sentiment -rather than topic- based categories. Hence, standard machine learning classification techniques, such as support vector machines (SVMs), can be applied to the entire documents themselves. Nevertheless, some researches presented a technique that it is easy to improve the accuracy, by integrating sentence-level subjectivity detection with document-level sentiment polarity.

As we know, in the machine learning approach, each classifier is trained using a collection of representative data. In contrast, the semantic-orientation approach does not require prior training; instead, it measures a word containing positive or negative sentiment. Each approach has its own benefits and drawbacks. For example, the machine learning approach tends to be more accurate, but the semantic-orientation approach has better generality. Recently, a new lexicon-enhanced method was accrued to generate a set of sentiment words based on a sentiment lexicon as a new feature dimension. It combines these sentiment features with content-free and content-specific features used in the existing machine-learning approach. In the evaluation stage, they showed that adding the new set of sentiment features can increase sentiment-classification performance.

The Internet and other communication technologies play a potentially disruptive role on the constraints imposed on social networks. These technologies reduce the overhead and cost for being introduced to new people regardless of geography, and help us stay in touch with those we know. Some have even gone so far as to call this ”the end of geography,” where the process of relationship formation becomes disentangled from distance altogether.

However, geography still plays an important role. The reason is because of the strong relationship between event detection and geographical location each user belongs. Twitter is a social networking website, which means that users need not be viewed in isolation, but instead can be viewed as part of a large network of other users, user groups, and user cliques. Moreover, users have some meta-data information, such as description, source location, friends, which means that the social network structure in Twitter can aid in finding users that are most likely to tweet about news belonging to a particular geographic location or region.

The rise of micro-blogging services spurred various applications to mine the data coming from those services. Many such applications could benefit from information about the location of users, but unfortunately location information is currently very sparse. The main problem is that less than 1 per cent of tweets are geo-tagged and information available from the location field in users’ profiles is unreliable at best. The benefits of mining those data promises new personalized information services, including local news summarized from tweets of nearby Twitter users, the targeting of regional advertisements, spreading business information to local customers, and novel location-based applications (e.g., Twitter-based earthquake detection, which can be faster than through traditional official channels).

There is a great number of geoinference using social networks One direction has produced approaches that claim to accurately locate the majority of posts within tens of kilometres of their true locations. Another method predicts the location of an individual from a sparse set of located users with peformance that exceeds IP-based geolocation. On the side, there is also a technique that predicts locations of Twitter users at different granularities, such as city, state, or time zone, using the content of their tweets and their tweeting behaviour.

Getting started the first section of this survey is the presentation of some techniques that aim to event detection. The survey detects both algorithms that are based on text categorization and frequency display methods. The second chapter deals with techniques of sentiment analysis. We give more weight on techniques that use machine learning. The last chapter unfolds methods for location identification.


\section{Event Detection}
Given a series of twitter posts, the goal of event detection is to extract a particular event by analysing the text or hashtag of a tweet. The process of event detection is not a lenient task as tweets stream in huge volumes and the level of noise is kept high \cite{petrovic2010streaming}. On the other hand, the huge volume of the stream allows the use of streaming algorithms, thus making event detection an accomplishable task \cite{petrovic2010streaming}. 
\par
The conventional approach for this problem is to represent the documents as term frequency vectors \cite{petrovic2010streaming}. When a new document arrives, it is compared to all previous ones, and if its similarity to a specific document called "centroid" is below a threshold, the new document is registered as a new event \cite{petrovic2010streaming}. Unfortunately, this simple approach doesn't perform well as the dimensionality of the data expands. 
\par
Twitter's users usually tweet about their personal life, personal opinions, taste in music etc., which makes the majority of tweets unsuitable for our purpose. Although, when a noteworthy event occurs, a lot of users tweet about it. The goal of event detection is to detect these events in an automatic fashion, keeping the noise (tweets that are not events) as low as possible \cite{petrovic2010streaming}.
\subsection{Strategies for event detection?}
As stated before, tweets stream in high rates and capture the pulse of the moment, therefore occupying us to process them right away \cite{Sankaranarayanan2009TwitterStandNI}. This is accomplished by using algorithms that are online in nature, meaning they can on inputs that are not known beforehand\cite{Sankaranarayanan2009TwitterStandNI}.
\par
Tweets are inherently noisy since most of them concern small user groups. In order to improve the results of event detection we first have to improve the input by extracting only useful information from noise \cite{Sankaranarayanan2009TwitterStandNI}. Even then, information that concerns us is still limited by the upper character limit set for each tweet, currently at 140 characters. Bad use of language, spelling errors and special puns also negatively affect the outcome of event detection.
\par
Twitter is an evolving platform where users come and go, and become friends and followers with each other. When designing event detection algorithms, one should keep in mind the above aspects and make the algorithms adapt well to new knowledge, otherwise watch them fail as time passes \cite{Sankaranarayanan2009TwitterStandNI}.
\subsubsection{Identifying users?}
An approach to event detection is the identification of users that usually tweet about news \cite{Sankaranarayanan2009TwitterStandNI}. A technique to do this is to manually identify these users by looking for the most common set of followers among them \cite{Sankaranarayanan2009TwitterStandNI}.
Another technique to do this is to assume that a user with a high amount of followers represents an influential event source into a social community \cite{cataldi2013personalized}. The added benefit of this method is that, not only we identify the users of interest but, we get an evaluation of a user's influence, meaning that an influential user can affect other users' tweet streams \cite{cataldi2013personalized}. This technique cascades as the follow-like relationships spread, making it comparable to the PageRank algorithm \cite{cataldi2013personalized}. An interesting fact to support the above claims is that studies have shown that 10\% of Twitter's users are responsible for 90\% of the tweets \cite{Sankaranarayanan2009TwitterStandNI}.








\section{Conclusions}
This paragraph will end the body of this sample document.
Remember that you might still have Acknowledgments or
Appendices; brief samples of these
follow.  There is still the Bibliography to deal with; and
we will make a disclaimer about that here: with the exception
of the reference to the \LaTeX\ book, the citations in
this paper are to articles which have nothing to
do with the present subject and are used as
examples only.
%\end{document}  % This is where a 'short' article might terminate

%ACKNOWLEDGMENTS are optional
\section{Acknowledgments}
This section is optional; it is a location for you
to acknowledge grants, funding, editing assistance and
what have you.  In the present case, for example, the
authors would like to thank Gerald Murray of ACM for
his help in codifying this \textit{Author's Guide}
and the \textbf{.cls} and \textbf{.tex} files that it describes.

%
% The following two commands are all you need in the
% initial runs of your .tex file to
% produce the bibliography for the citations in your paper.
\bibliographystyle{abbrv}
\bibliography{sigproc}  % sigproc.bib is the name of the Bibliography in this case
\nocite{*} %auto einai gia na bgazei to reference xwris na xreiazetai na to kanoume cite kapou.
% You must have a proper ".bib" file
%  and remember to run:
% latex bibtex latex latex
% to resolve all references
%
% ACM needs 'a single self-contained file'!
%
%APPENDICES are optional
%\balancecolumns
\appendix
%Appendix A
\section{Headings in Appendices}
The rules about hierarchical headings discussed above for
the body of the article are different in the appendices.
In the \textbf{appendix} environment, the command
\textbf{section} is used to
indicate the start of each Appendix, with alphabetic order
designation (i.e. the first is A, the second B, etc.) and
a title (if you include one).  So, if you need
hierarchical structure
\textit{within} an Appendix, start with \textbf{subsection} as the
highest level. Here is an outline of the body of this
document in Appendix-appropriate form:
\subsection{Introduction}
\subsection{The Body of the Paper}
\subsubsection{Type Changes and  Special Characters}
\subsubsection{Math Equations}
\paragraph{Inline (In-text) Equations}
\paragraph{Display Equations}
\subsubsection{Citations}
\subsubsection{Tables}
\subsubsection{Figures}
\subsubsection{Theorem-like Constructs}
\subsubsection*{A Caveat for the \TeX\ Expert}
\subsection{Conclusions}
\subsection{Acknowledgments}
\subsection{Additional Authors}
This section is inserted by \LaTeX; you do not insert it.
You just add the names and information in the
\texttt{{\char'134}additionalauthors} command at the start
of the document.
\subsection{References}
Generated by bibtex from your ~.bib file.  Run latex,
then bibtex, then latex twice (to resolve references)
to create the ~.bbl file.  Insert that ~.bbl file into
the .tex source file and comment out
the command \texttt{{\char'134}thebibliography}.
% This next section command marks the start of
% Appendix B, and does not continue the present hierarchy
\section{More Help for the Hardy}
The sig-alternate.cls file itself is chock-full of succinct
and helpful comments.  If you consider yourself a moderately
experienced to expert user of \LaTeX, you may find reading
it useful but please remember not to change it.
%\balancecolumns % GM June 2007
% That's all folks!
\end{document}
